\documentclass{article}
\bibliographystyle{plain}

\newcommand{\EV}{\mathrm{E}}
\newcommand{\Var}{\mathrm{Var}}
\newcommand{\aRule}{\begin{center} \rule{5in}{1mm} \end{center}}

\title{Generalized Boosted Models:\\A guide to the gbm package} \author{Greg Ridgeway}

%\VignetteEngine{knitr::knitr}
%\VignetteIndexEntry{Generalized Boosted Models: A guide to the gbm package}

\newcommand{\mathgbf}[1]{{\mbox{\boldmath$#1$\unboldmath}}}

\usepackage{Sweave}
\begin{document}
\Sconcordance{concordance:gbm.tex:gbm.Rnw:%
1 14 1 1 0 4 1 1 5 343 1 1 64 295 1}


\maketitle


Boosting takes on various forms with different programs using
different loss functions, different base models, and different
optimization schemes. The gbm package takes the approach described in
\cite{Friedman:2001} and \cite{Friedman:2002}. Some of the terminology
differs, mostly due to an effort to cast boosting terms into more
standard statistical terminology (e.g. deviance). In addition, the gbm
package implements boosting for models commonly used in statistics but
not commonly associated with boosting. The Cox proportional hazard
model, for example, is an incredibly useful model and the boosting
framework applies quite readily with only slight modification
\cite{Ridgeway:1999}. Also some algorithms implemented in the gbm
package differ from the standard implementation. The AdaBoost
algorithm \cite{FreundSchapire:1997} has a particular loss function
and a particular optimization algorithm associated with it. The gbm
implementation of AdaBoost adopts AdaBoost's exponential loss function
(its bound on misclassification rate) but uses Friedman's gradient
descent algorithm rather than the original one proposed. So the main
purposes of this document is to spell out in detail what the gbm
package implements.

\section{Gradient boosting}

This section essentially presents the derivation of boosting described
in \cite{Friedman:2001}. The gbm package also adopts the stochastic
gradient boosting strategy, a small but important tweak on the basic
algorithm, described in \cite{Friedman:2002}.

\subsection{Friedman's gradient boosting machine} \label{sec:GradientBoostingMachine}

\begin{figure} 
  \aRule Initialize $\hat f(\mathbf{x})$ to be a constant, $\hat f(\mathbf{x}) = \arg \min_{\rho} \sum_{i=1}^N \Psi(y_i,\rho)$. \\
  For $t$ in $1,\ldots,T$ do
\begin{enumerate} 
\item Compute the negative gradient as the working response
    \begin{equation}
    z_i = -\frac{\partial}{\partial f(\mathbf{x}_i)} \Psi(y_i,f(\mathbf{x}_i)) \mbox{\Huge $|$}_{f(\mathbf{x}_i)=\hat f(\mathbf{x}_i)}
    \end{equation}
\item Fit a regression model, $g(\mathbf{x})$, predicting $z_i$ from the covariates $\mathbf{x}_i$. \item Choose a gradient descent step size as
    \begin{equation}
    \rho = \arg \min_{\rho} \sum_{i=1}^N \Psi(y_i,\hat f(\mathbf{x}_i)+\rho g(\mathbf{x}_i))
    \end{equation}
\item Update the estimate of $f(\mathbf{x})$ as
    \begin{equation}
    \hat f(\mathbf{x}) \leftarrow \hat f(\mathbf{x}) + \rho g(\mathbf{x})
    \end{equation}
\end{enumerate} \aRule \caption{Friedman's Gradient Boost algorithm} \label{fig:GradientBoost} \end{figure}

Friedman (2001) and the companion paper Friedman (2002) extended the
work of Friedman, Hastie, and Tibshirani (2000) and laid the ground
work for a new generation of boosting algorithms. Using the connection
between boosting and optimization, this new work proposes the Gradient
Boosting Machine.

In any function estimation problem we wish to find a regression
function, $\hat f(\mathbf{x})$, that minimizes the expectation of some
loss function, $\Psi(y,f)$, as shown in
(\ref{NonparametricRegression1}).

\begin{eqnarray}
\hspace{0.5in}
\hat f(\mathbf{x}) &=& \arg \min_{f(\mathbf{x})} \EV_{y,\mathbf{x}} \Psi(y,f(\mathbf{x})) \nonumber \\ \label{NonparametricRegression1}
&=& \arg \min_{f(\mathbf{x})} \EV_x \left[ \EV_{y|\mathbf{x}} \Psi(y,f(\mathbf{x})) \Big| \mathbf{x} \right]
\end{eqnarray}

We will focus on finding estimates of $f(\mathbf{x})$ such that
\begin{equation}
\label{NonparametricRegression2}
\hspace{0.5in}
\hat f(\mathbf{x}) = \arg \min_{f(\mathbf{x})} \EV_{y|\mathbf{x}} \left[ \Psi(y,f(\mathbf{x}))|\mathbf{x} \right]
\end{equation}

Parametric regression models assume that $f(\mathbf{x})$ is a function
with a finite number of parameters, $\beta$, and estimates them by
selecting those values that minimize a loss function (e.g. squared
error loss) over a training sample of $N$ observations on
$(y,\mathbf{x})$ pairs as in (\ref{eq:Friedman1}).
\begin{equation}
\label{eq:Friedman1}
\hspace{0.5in}
\hat\beta = \arg \min_{\beta} \sum_{i=1}^N \Psi(y_i,f(\mathbf{x}_i;\beta))
\end{equation}

When we wish to estimate $f(\mathbf{x})$ non-parametrically the task
becomes more difficult. Again we can proceed similarly to
\cite{FHT:2000} and modify our current estimate of $f(\mathbf{x})$ by
adding a new function $f(\mathbf{x})$ in a greedy fashion. Letting
$f_i = f(\mathbf{x}_i)$, we see that we want to decrease the $N$
dimensional function
\begin{eqnarray}
\label{EQ:Friedman2}
\hspace{0.5in}
J(\mathbf{f}) &=& \sum_{i=1}^N \Psi(y_i,f(\mathbf{x}_i)) \nonumber \\
                          &=& \sum_{i=1}^N \Psi(y_i,F_i).
\end{eqnarray}
The negative gradient of $J(\mathbf{f})$ indicates the direction of
the locally greatest decrease in $J(\mathbf{f})$.  Gradient descent
would then have us modify $\mathbf{f}$ as
\begin{equation}
\label{eq:Friedman3}
\hspace{0.5in}
\hat \mathbf{f} \leftarrow \hat \mathbf{f} - \rho \nabla J(\mathbf{f})
\end{equation}
where $\rho$ is the size of the step along the direction of greatest
descent. Clearly, this step alone is far from our desired goal. First,
it only fits $f$ at values of $\mathbf{x}$ for which we have
observations.  Second, it does not take into account that observations
with similar $\mathbf{x}$ are likely to have similar values of
$f(\mathbf{x})$. Both these problems would have disastrous effects on
generalization error. However, Friedman suggests selecting a class of
functions that use the covariate information to approximate the
gradient, usually a regression tree. This line of reasoning produces
his Gradient Boosting algorithm shown in
Figure~\ref{fig:GradientBoost}. At each iteration the algorithm
determines the direction, the gradient, in which it needs to improve
the fit to the data and selects a particular model from the allowable
class of functions that is in most agreement with the direction. In
the case of squared-error loss, $\Psi(y_i,f(\mathbf{x}_i)) =
\sum_{i=1}^N (y_i-f(\mathbf{x}_i))^2$, this algorithm corresponds
exactly to residual fitting.

There are various ways to extend and improve upon the basic framework
suggested in Figure~\ref{fig:GradientBoost}. For example, Friedman
(2001) substituted several choices for $\Psi$ to develop new boosting
algorithms for robust regression with least absolute deviation and
Huber loss functions. Friedman (2002) showed that a simple subsampling
trick can greatly improve predictive performance while simultaneously
reduce computation time. Section~\ref{GBMModifications} discusses some
of these modifications.

\section{Improving boosting methods using control of the learning rate, sub-sampling, and a decomposition for interpretation} \label{GBMModifications}

This section explores the variations of the previous algorithms that
have the potential to improve their predictive performance and
interpretability. In particular, by controlling the optimization speed
or learning rate, introducing low-variance regression methods, and
applying ideas from robust regression we can produce non-parametric
regression procedures with many desirable properties. As a by-product
some of these modifications lead directly into implementations for
learning from massive datasets. All these methods take advantage of
the general form of boosting
\begin{equation} 
\hat f(\mathbf{x}) \leftarrow \hat f(\mathbf{x}) + \EV(z(y,\hat f(\mathbf{x}))|\mathbf{x}). 
\end{equation}
So far we have taken advantage of this form only by substituting in
our favorite regression procedure for $\EV_w(z|\mathbf{x})$. I will
discuss some modifications to estimating $\EV_w(z|\mathbf{x})$ that
have the potential to improve our algorithm.

\subsection{Decreasing the learning rate}
As several authors have phrased slightly differently, ``...boosting,
whatever flavor, seldom seems to overfit, no matter how many terms are
included in the additive expansion''. This is not true as the
discussion to \cite{FHT:2000} points out.

In the update step of any boosting algorithm we can introduce a
learning rate to dampen the proposed move.
\begin{equation} 
\label{eq:shrinkage} 
\hat f(\mathbf{x}) \leftarrow \hat f(\mathbf{x}) + \lambda \EV(z(y,\hat f(\mathbf{x}))|\mathbf{x}). 
\end{equation} 
By multiplying the gradient step by $\lambda$ as in
equation~\ref{eq:shrinkage} we have control on the rate at which the
boosting algorithm descends the error surface (or ascends the
likelihood surface). When $\lambda=1$ we return to performing full
gradient steps. Friedman (2001) relates the learning rate to
regularization through shrinkage.

The optimal number of iterations, $T$, and the learning rate,
$\lambda$, depend on each other. In practice I set $\lambda$ to be as
small as possible and then select $T$ by cross-validation. Performance
is best when $\lambda$ is as small as possible, with
decreasing marginal utility for smaller and smaller $\lambda$. Slower
learning rates do not necessarily scale the number of optimal
iterations. That is, when $\lambda=1.0$ and the optimal $T$ is 100
iterations, it does {\it not} necessarily imply that when $\lambda=0.1$
the optimal $T$ is 1000 iterations.

\subsection{Variance reduction using subsampling}

Friedman (2002) proposed the stochastic gradient boosting algorithm
that simply samples uniformly without replacement from the dataset
before estimating the next gradient step. He found that this
additional step greatly improved performance. We estimate the
regression $\EV(z(y,\hat f(\mathbf{x}))|\mathbf{x})$ using a random
subsample of the dataset.

\subsection{ANOVA decomposition}

Certain function approximation methods are decomposable in terms of a
``functional ANOVA decomposition''. That is a function is decomposable
as
\begin{equation} 
\label{ANOVAdecomp} 
f(\mathbf{x}) = \sum_j f_j(x_j) + \sum_{jk} f_{jk}(x_j,x_k) + \sum_{jk\ell} f_{jk\ell}(x_j,x_k,x_\ell) + \cdots. 
\end{equation}
This applies to boosted trees. Regression stumps (one split decision
trees) depend on only one variable and fall into the first term of
\ref{ANOVAdecomp}. Trees with two splits fall into the second term of
\ref{ANOVAdecomp} and so on. By restricting the depth of the trees
produced on each boosting iteration we can control the order of
approximation. Often additive components are sufficient to approximate
a multivariate function well, generalized additive models, the
na\"{\i}ve Bayes classifier, and boosted stumps for example. When the
approximation is restricted to a first order we can also produce plots
of $x_j$ versus $f_j(x_j)$ to demonstrate how changes in $x_j$ might
affect changes in the response variable.

\subsection{Relative influence}
Friedman (2001) also develops an extension of a variable's ``relative
influence'' for boosted estimates. For tree based methods the
approximate relative influence of a variable $x_j$ is
\begin{equation} 
\label{RelInfluence} 
\hspace{0.5in} 
\hat J_j^2 = \hspace{-0.1in}\sum_{\mathrm{splits~on~}x_j}\hspace{-0.2in}I_t^2
\end{equation}
where $I_t^2$ is the empirical improvement by splitting on $x_j$ at
that point. Friedman's extension to boosted models is to average the
relative influence of variable $x_j$ across all the trees generated by
the boosting algorithm.

\begin{figure}
\aRule
Select
\begin{itemize}
\item a loss function (\texttt{distribution})
\item the number of iterations, $T$ (\texttt{n.trees})
\item the depth of each tree, $K$ (\texttt{interaction.depth})
\item the shrinkage (or learning rate) parameter, $\lambda$
  (\texttt{shrinkage})
\item the subsampling rate, $p$ (\texttt{bag.fraction})
\end{itemize}
Initialize $\hat f(\mathbf{x})$ to be a constant, $\hat f(\mathbf{x}) = \arg \min_{\rho} \sum_{i=1}^N \Psi(y_i,\rho)$ \\
For $t$ in $1,\ldots,T$ do
\begin{enumerate}
\item Compute the negative gradient as the working response
    \begin{equation}
    z_i = -\frac{\partial}{\partial f(\mathbf{x}_i)} \Psi(y_i,f(\mathbf{x}_i)) \mbox{\Huge $|$}_{f(\mathbf{x}_i)=\hat f(\mathbf{x}_i)}
    \end{equation}
\item Randomly select $p\times N$ cases from the dataset
\item Fit a regression tree with $K$ terminal nodes,
  $g(\mathbf{x})=\EV(z|\mathbf{x})$. This tree is fit using only those
  randomly selected observations
\item Compute the optimal terminal node predictions, $\rho_1,\ldots,\rho_K$, as
    \begin{equation}
    \rho_k = \arg \min_{\rho} \sum_{\mathbf{x}_i\in S_k} \Psi(y_i,\hat f(\mathbf{x}_i)+\rho)
    \end{equation}
    where $S_k$ is the set of $\mathbf{x}$s that define terminal node
    $k$. Again this step uses only the randomly selected observations.
\item Update $\hat f(\mathbf{x})$ as
    \begin{equation}
    \hat f(\mathbf{x}) \leftarrow \hat f(\mathbf{x}) + \lambda\rho_{k(\mathbf{x})}
    \end{equation}
    where $k(\mathbf{x})$ indicates the index of the terminal node
    into which an observation with features $\mathbf{x}$ would fall.
\end{enumerate}
\aRule
\caption{Boosting as implemented in \texttt{gbm()}}
\label{fig:gbm}
\end{figure}

\section{Common user options}

This section discusses the options to gbm that most users will need to
change or tune.

\subsection{Loss function}

The first and foremost choice is \texttt{distribution}. This should be
easily dictated by the application. For most classification problems
either \texttt{bernoulli} or \texttt{adaboost} will be appropriate,
the former being recommended. For continuous outcomes the choices are
\texttt{gaussian} (for minimizing squared error), \texttt{laplace}
(for minimizing absolute error), and quantile regression (for
estimating percentiles of the conditional distribution of the
outcome). Censored survival outcomes should require
\texttt{coxph}. Count outcomes may use \texttt{poisson} although one
might also consider \texttt{gaussian} or \texttt{laplace} depending on
the analytical goals.

\subsection{The relationship between shrinkage and number of iterations}

The issues that most new users of gbm struggle with are the choice of
\texttt{n.trees} and \texttt{shrinkage}. It is important to know that
smaller values of \texttt{shrinkage} (almost) always give improved
predictive performance. That is, setting \texttt{shrinkage=0.001} will
almost certainly result in a model with better out-of-sample
predictive performance than setting \texttt{shrinkage=0.01}. However,
there are computational costs, both storage and CPU time, associated
with setting \texttt{shrinkage} to be low. The model with
\texttt{shrinkage=0.001} will likely require ten times as many
iterations as the model with \texttt{shrinkage=0.01}, increasing
storage and computation time by a factor of
10. Figure~\ref{fig:shrinkViters} shows the relationship between
predictive performance, the number of iterations, and the shrinkage
parameter. Note that the increase in the optimal number of iterations
between two choices for shrinkage is roughly equal to the ratio of the
shrinkage parameters. It is generally the case that for small
shrinkage parameters, 0.001 for example, there is a fairly long
plateau in which predictive performance is at its best. My rule of
thumb is to set \texttt{shrinkage} as small as possible while still
being able to fit the model in a reasonable amount of time and
storage. I usually aim for 3,000 to 10,000 iterations with shrinkage
rates between 0.01 and 0.001.

